\documentclass[12pt,letterpaper,titlepage,oneside,draft]{article}
\usepackage[utf8]{inputenc}
\usepackage{amsmath}
\usepackage{amsfonts}
\usepackage{amssymb}

\title{The Genius of Hollywood}
\author{
	Zachary Bricker  \\
	zbricker@my.harrisburgu.edu
	\and
	Douglas Bahr Rumbaugh \\
	dbrumbaugh@my.harrisburgu.edu
	\and 
	Chad Van Chu \\
	cvchu@my.harrisburgu.edu
	\\
	\\
	Harrisburg University of Science and Technology
}

\date{}

\begin{document}
\maketitle
\section*{Abstract}
\tableofcontents
\pagebreak


\section{Introduction}
In recent years, it has become fashionable to possess a high IQ. It is, then, no big surprise that recent years have also seen a massive surge in public figures reported IQ scores. Of particular interest to the authors are the claims of those in the entertainment industry.

One needs not look terribly hard in order to stumble across news articles, both in traditional media and the tabloid press, claiming to posses a list of the smartest celebrities. Interestingly enough, in the authors' experience, such articles rarely provide evidence for their claims.

As such, the authors determined that it would be an interesting exercise to examine the implications of the truth of these reported IQ scores. There are 3,000 seats at the academy awards--so what is the probability of random chance alone bringing such an assortment of geniuses into the same room? That is the question this study hopes to answer.
\section{Review of Literature}

\section{Methods}
In order to obtain the data for this analysis, reported IQ scores for celebrities were aggregated from several Internet sources. Unsurprisingly, for several celebrities, there were discrepancies between IQs reported from source to source. These discrepancies are discussed in further detail in the Discussion section. For the primary analysis though, a single IQ value per person was required. In these cases, the most frequently occurring reported IQ was selected, under the assumption that the most frequently repeated score was the one with the most popular acceptance.

Using the table of IQ scores, each individual was assigned a probability. In order to obtain the probability of all of these individuals being in the 3,000 seats at the Academy Awards, the product of all individual probabilities was taken, and this product was multiplied by 3,000. 
\section{Results}

\section{Discussion}

\end{document}